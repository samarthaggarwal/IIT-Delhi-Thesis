% \noindent KEYWORDS: \hspace*{0.5em} \parbox[t]{4.4in}{\LaTeX ; Thesis;
%   Style files; Format.}

% \vspace*{24pt}

% \noindent A \LaTeX\ class along with a simple template thesis are
% provided here.  These can be used to easily write a thesis suitable
% for submission at IIT-Delhi.  The class provides options to format
% PhD, MS, M.Tech.\ and B.Tech.\ thesis.  It also allows one to write a
% synopsis using the same class file.  Also provided is a BIB\TeX\ style
% file that formats all bibliography entries as per the IITD format.

% The formatting is as (as far as the author is aware) per the current
% institute guidelines.

%%%%% Abstract = Summary of the entire project %%%%%

%%%%%%%%%%%%%%%%%%%%%%%%%%%%%%%%%%%%%%%%%%%%%%%%%%%%%%%%%%%%%%%%%%
\begin{comment}

  Skelaton

What is OpenIE - 1 line
list a few systems
problems with recent systems
new benchmark was needed
so we made carb, how is carb different
problems with openie models
imojie, mention perf improvements
errors made by imojie, add error analysis
mention mlil paper

\end{comment}
%%%%%%%%%%%%%%%%%%%%%%%%%%%%%%%%%%%%%%%%%%%%%%%%%%%%%%%%%%%%%%%%%%

Open Information Extraction refers to the task of obtaining relation tuples from a sentence. For eg. the sentence ``Donald Trump is the president of United States.'' yields (Donald Trump ; is the president of ; United States) as its OpenIE tuple.

The Open IE paradigm is a useful intermediary for a variety of down-stream tasks such as sentence similarity, event schema induction, text comprehension, knowledge base completion, and more. There have been several attempts at building OpenIE systems that explored rule-based such as OllIE, OpenIE-4 and OpenIE-5. Another wave of OpenIE systems that followed, comprised of neural approaches such as RnnOIE and \citet{cui&al18}. However, the existing openie systems suffer from a wide range of problems. The rule-based systems suffered from cascading errors from a large number of components in succession. The existing neural OpenIE systems, although were able to solve some of these issues to a certain extent, were still far from ideal. Infact, they introduced other problems such as redundancy in their outputs. Together these factors solicit an OpenIE system that is able to overcome the issues pertaining to OpenIE.

Although human inspection revealed that the existing systems were not upto the mark, yet these systems scored high on the existing state-of-the-art OpenIE benchmarks such as \cite{OIE2016}. This means that the existing benchmarks do not correlate well with how humans evaluate OpenIE. In response, we contribute CaRB \cite{bhardwaj&al19}, with a high-quality crowdsourced gold dataset and intuitive evaluation policies that correlate well with human judgement of OpenIE. CaRB establishes itself as the new state of the art OpenIE benchmark.

CaRB evaluation of the \citet{cui&al18}, then state of the art OpenIE systems, confirms its inept performance. We contribute IMoJIE \cite{kolluru&al20}, a neural OpenIE model that outperforms the previous state of the art by about 18 F1 points. It reduces the redundancy in output extractions significantly. Along with it, IMoJIE also presents a novel approach that can be used to generation high-quality training data from multiple low quality datasets.

Although IMoJIE improves the quality of OpenIE tuples significantly, this improvement comes at the cost of speed of extraction. We design a MLIL architecture to overcome the issue of speed of extraction and also obtain further performance nudges from it. This approach also yields a coordination analyzer that significantly improves the yield of the MLIL model.

In the end, we analyse the milestones covered in the world of OpenIE and contribute some ideas for future research.