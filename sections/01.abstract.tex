%%%%%%%%%%%%%%%%%%%%%%%%%%%%%%%%%%%%%%%%%%%%%%%%%%%%%%%%%%%%%%%%%%
\begin{comment}

  Skelaton

What is OpenIE - 1 line
list a few systems
problems with recent systems
new benchmark was needed
so we made carb, how is carb different
problems with openie models
imojie, mention perf improvements
errors made by imojie, add error analysis
mention mlil paper

\end{comment}
%%%%%%%%%%%%%%%%%%%%%%%%%%%%%%%%%%%%%%%%%%%%%%%%%%%%%%%%%%%%%%%%%%

Open Information Extraction refers to the task of obtaining relation tuples from a sentence. For eg. the sentence \verb|``Donald Trump is the president of United States.''| yields \verb|(Donald Trump ; is the president of ; United States)| as its OpenIE tuple.

The Open IE paradigm is a useful intermediary for a variety of down-stream tasks such as sentence similarity, event schema induction, text comprehension, knowledge base completion, and more. There have been several attempts at building OpenIE systems that explored rule-based such as ClausIE, OpenIE-4 and OpenIE-5. Another wave of OpenIE systems that followed, comprised of neural approaches such as RnnOIE and \citet{cui&al18}. However, the rule-based systems suffered from cascading errors from a large number of components in succession. The existing neural OpenIE systems, although have improved, yet are far from ideal. Infact, they introduced other problems such as redundancy in their outputs. Together these factors solicit an OpenIE system that is able to overcome the issues pertaining to OpenIE.

Human inspection revealed that the existing systems were not upto the mark, yet these systems scored high on the existing OpenIE benchmarks such as OIE2016 \citep{OIE2016}. This meant that the existing benchmarks do not correlate well with how humans evaluate OpenIE. In response, we contribute CaRB \citep{bhardwaj&al19}, with a high-quality crowdsourced gold dataset and intuitive evaluation policies that correlate well with human judgement of OpenIE. CaRB establishes itself as the new state of the art OpenIE benchmark.

CaRB evaluation of the \citet{cui&al18}, then state of the art OpenIE systems, confirms its inept performance. We contribute IMoJIE \citep{kolluru&al20}, a neural OpenIE model that outperforms the previous state of the art by about 18 F1 points. It reduces the redundancy in output extractions significantly. Along with it, IMoJIE also presents a novel approach that can be used to generation high-quality training data from multiple low quality datasets.

Although IMoJIE improves the quality of OpenIE tuples significantly, this improvement comes at the cost of speed of extraction. We design a MLIL architecture to overcome the issue of speed of extraction and also obtain further performance nudges from it. The MLIL approach also yields a coordination analyzer that establishes itself as a state-of-the-art in the task and also increases the yield of the MLIL based OpenIE to 1.7x.

We also analyse OpenIE milestones and contribute some ideas for future research.

%%%%%%%%%%%%%%%%%%%%%%%%%%%%%%%%%%%% MOVE TO ABSTRACT %%%%%%%%%%%%%%%%%%%%%%%%%%%%%%%%%%%%
% While traditional systems for Open Information Extraction were statistical and rule-based, recently neural models have been introduced for the task. Our work builds upon CopyAttention, a sequence generation OpenIE model \cite{cui&al18}. Our analysis reveals that CopyAttention produces a constant number of extractions per sentence, and its extracted tuples often express redundant information.

% We present \shortname, an extension to CopyAttention, which produces the next extraction conditioned on all previously extracted tuples. This approach overcomes both shortcomings of CopyAttention, resulting in a variable number of diverse extractions per sentence. We train \shortname{} on training data bootstrapped from extractions of several non-neural systems, which have been automatically filtered to reduce redundancy and noise.  \shortname{} outperforms CopyAttention by about 18 F1 pts, and a BERT-based strong baseline by 2 F1 pts, establishing a new state of the art for the task.

%%%%%%%%%%%%%%%%%%%%%%%%%%%%%%%%%%%%%%%%%%%%%%%%%%%%%%%%%%%%%%%%%%%%%%%%%%%%%%%%%%%%%%%%%%
